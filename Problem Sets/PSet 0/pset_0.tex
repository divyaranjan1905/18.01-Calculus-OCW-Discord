\documentclass{article}
\title{\textbf{Problem Set 0 : Prerequisites}}
\author{18.01 Single Variable Calculus}
\date{19th September, 2021}

\input{preamble.tex}


\begin{document}
\maketitle

\section{Problems}
\subsection{Functions and their Graphs}

\noindent
1. For a curve to be \textit{symmetric about the x-axis}, the point $(x,y)$ must lie on the curve if and only if the point $(x,-y)$ lies on the curve. Explain why a curve that is symmetric about the $x$-axis is not the graph of a function, unless the function is $y=0$.

\noindent
\vspace{2.5mm}

2. Graph this piecewise function:
\[ F(x) = \begin{cases}
	   4-x^{2x}, &  x \leq 1 \\
	   x^{2}+2x, &  x > 1 
	  \end{cases}
\]

\subsection{Trigonometry}
3. Apply the formula for $\cos(A-B)$ to the identity $\sin \theta = \cos \qty(\frac{\pi}{2} - \theta )$ to obtain the addition formula for $\sin (A+B)$.

\noindent
\vspace{2.5mm}

4. Graph the function $\cos \qty(x - \frac{\pi}{2})$ and find its period. 

\subsection{Transcendental Functions ($e$ and $\ln{x}$}

\noindent
\vspace{2.5mm}
Solve for $t$:
5.$ \ln{\qty(\frac{t}{t-1})} = 2$

\noindent
\vspace{2.5mm}

6. $e^{2t}-3e^{t}=0$

\subsection{The Binomial Theorem}

\noindent
\vspace{2.5mm}

7. Give an algebraic and combinatorial proof of the identity\footnote{Here algebraic proof means just using the binomial theorem and \textit{algebraically} proving the given identity. But the combinatorial proof is one where you prove the identity using combinatorial arguments, i.e. by coming up with several ways of counting.} 
\[ \binom{n}{k} \binom{n-2}{k-2} = \binom{n}{k} \binom{k}{2} \]

\noindent
\vspace{2.5mm}

8. For $n \geq 1$, verify that
\[ 1^{2} + 3^{2} + 5^{2} \dots + (2n+1)^{2} = \binom{2n+1}{3} \]

\subsection{Complex Numbers}
\noindent
\vspace{2.5mm}

9. Solve for real numbers $x$ and $y$:
\[ \qty(\frac{1+i}{1-i})^{2} + \frac{1}{x+iy} = 1+i \]

\noindent
\vspace{2.5mm}

10. Find all solutions of the equation $x^{4}+4x^{2}+16=0$.







\end{document}
